\documentclass[10pt]{beamer}

\usepackage[utf8]{inputenc}
\usepackage[T2A]{fontenc}
\usepackage[russian]{babel}
\usepackage{hyperref}
\usepackage{amsmath}
%\usepackage[footnotes,oglav,spisok,boldsect,eqwhole,kursrab,hyperprint]{project1}
\usetheme{Copenhagen}
\useoutertheme{default}
\usecolortheme{sidebartab}
%\usefonttheme{serif}
\useoutertheme[]{miniframes}
\usepackage{graphicx}
\usepackage{lipsum}
%\usepackage{rumathgrk1}
\usepackage{booktabs}

%\usepackage{glonti}
\defbeamertemplate*{footline}{Warsaw} {%
\leavevmode%
\hbox{%
\begin{beamercolorbox}[wd=.5\paperwidth,ht=2.5ex,dp=1.125ex,leftskip=.3cm,rightskip=.3cm]{author in head/foot}%
\insertframenumber{}%
\hfill\insertshortauthor
\end{beamercolorbox}%
\begin{beamercolorbox}[wd=.5\paperwidth,ht=2.5ex,dp=1.125ex,leftskip=.3cm,rightskip=.3cm]{title in head/foot}%
\usebeamerfont{title in head/foot}\insertshorttitle
\end{beamercolorbox}
}%
\vskip0pt%
}


% \setbeamersize
% {
%     text margin left=0.8cm,
%     text margin right=0.8cm
% }
\renewcommand{\thempfootnote}{\arabic{mpfootnote}}
\title{\textbf{Восстановление человеком исходной позы после толчка \\
Reversion of initial posture by a person after a push}}

\author{\textbf{Романов Андрей Владимирович}}
\institute{\textbf{МГУ им. М.В. Ломоносова}\\\textbf{Механико-математический факультет} 
\\ \textbf{Кафедра прикладной механики и управления}
\\ \textbf{Научный руководитель: Кручинин П.А.}}
\date{\today}



\begin{document}

\maketitle

\begin{frame}{Описание задачи}
	\begin{figure}[h!]
		\begin{center}
			\begin{minipage}[h]{0.33\linewidth}
				\includegraphics[width=1\linewidth]{images/human.png}
				\caption{Схематическое изображение толкателя
					и положения испытуемого на стабилоплатформе}
			\end{minipage}
			\hfill
			\begin{minipage}[h]{0.66\linewidth}
				\includegraphics[width=1\linewidth]{images/Pushes.png}
				{\footnotesize
					\caption{Отклонение сагиттальной координаты при различных по силе толчках (данные предоставлены сотрудниками ИМБП РАН) }
				}
			\end{minipage}
		\end{center}
	\end{figure}
\end{frame}

\begin{frame}{Задача быстродействия}
	\begin{columns}
		\column{0.55\textwidth}
		\begin{figure}[h!]
			\includegraphics[width=1\linewidth]{images/stabilos.png}
			\caption{Характерный вид сагиттальной стабилограммы при наклоне при выполнении теста со ступенчатым
			 воздействием}
		\end{figure}
		\column{0.6\textwidth}
		% \column{\dimexpr\paperwidth-2pt}
		В работе рассматриваются возможные алгоритмы управления изменением
		позы человека, основанные на решении задачи оптимального быстродействия,
		которые можно было бы использовать для возвращения человека в исходную 
		вертикальную позу. В качестве математической модели используется модель
		«перевернутого маятника». Это решение предлагается использовать для
		оценки эффективности управления человеком при возвращении в
		вертикальную позу, путем сравнения времени реального процесса с полученным
		эталонным решением оптимальной задачи.
	\end{columns}
\end{frame}

\begin{frame}{Математическая модель}
	Рассматривается задача возвращения в исходную позу после завершения толчка
	\begin{columns}
		\column{0.62\textwidth}
		\begin{figure}[h!]
			\includegraphics[width=0.6\linewidth]{images_pres/body_1.png}
			\caption{Модель перевернутого маятника}
		\end{figure}
		\column{0.5\textwidth}

		\[
			J\ddot{\varphi}=m_Tgl\varphi+M
		\]
		\[
			\varphi(0)=\varphi_0,\, \dot{\varphi}(0)=\omega_0
		\]
		\[
			\varphi(t)=\varphi_k,\, \dot{\varphi}(t_k)=0
		\]
		\[
			M(0)=M(t_k)=-m_Tgl\varphi_k
		\]
		\[
			U^-\leq\dot{M}\leq U^+
		\]
	\end{columns}
\end{frame}
\begin{frame}{Решение задачи быстродействия}
	В прошлом году решалась задача быстродействия

	Система разбивается на 3 этапа, с чередованием знака управления

	Решение сводится к отысканию корней полинома для нахождения времени возвращения в вертикальную позицию.

	$\theta - \text{угол отклонения от вертикали}$

	$\omega - \text{угловая скорость тела}$

	$m - \text{момент, возникающий в голеностопном суставе}$


	\begin{columns}
		\column{0.5\textwidth}
		\begin{equation}\label{basesystem}
			\left\{ {\begin{aligned}
						 & \theta^{'} = \omega , \hfill   \\
						 & \omega^{'} = \theta+m , \hfill \\
						 & m^{'} = u . \hfill             \\
					\end{aligned}} \right.
		\end{equation}
		\column{0.5\textwidth}
		\[
			u=
			\begin{cases}
				-u_{max} \\
				+u_{max}
			\end{cases}
		\]\\*
	\end{columns}
	
\[
    \tau=\frac{t}{t_\ast},\ t_\ast=\sqrt{\frac{J}{m_Tgl}},\ \varphi_\ast=\varphi_0-\varphi_k
\]
	\[
		\theta(0)=1;\ \theta^{'}(0)=\frac{t_\ast}{\varphi_\ast}\omega_0=\Omega_0;\ m(0)=0
	\]
	\[
		\theta(\tau_f)=0;\ \theta^{'}(\tau_f)=0;\ m(\tau_f)=0
	\]
\end{frame}
\begin{frame}{Решение задачи быстродействия}
	Запишем функцию Понтрягина
\[
    H(\psi(t),y(t),u(t))=\psi_1\cdot\omega+\psi_2\cdot(\theta+m)+\psi_3\cdot u
\]

\begin{equation} \label{7}
    \left\{ {\begin{aligned}
                 & \psi^{'}_1=  - \frac{{\partial H}}{{\partial \theta}} = - \psi_2\hfill  \\
                 & \psi^{'}_2=  - \frac{{\partial H}}{{\partial \omega }} = - \psi_1\hfill \\
                 & \psi^{'}_3=  - \frac{{\partial H}}{{\partial m }} = - \psi_2 \hfill     \\
            \end{aligned}} \right.
\end{equation}
При $\psi_3\equiv0$ следует, что $\psi_2\equiv0$ и $\psi_1\equiv0$ следовательно особого управления нет.

Тогда для условия максимизации функции Понтрягина
\[
    u=
    \begin{cases}
        -u_{max}, & \text{при $\psi_3<0$}          \\
        +u_{max}, & \text{при $\psi_3\geqslant 0$}
    \end{cases}
\]\\*
\end{frame}
\begin{frame}{Решение задачи быстродействия}
Решая систему \eqref{7}, получим
\[
    \left\{ {\begin{aligned}
                 & \psi_1 = -C_1e^\tau+C_2e^{-\tau}+C_3, \hfill  \\
                 & \psi_2 = C_1e^\tau+C_2e^{-\tau} , \hfill      \\
                 & \psi_3 = -C_1e^\tau+C_2e^{-\tau}+C_3 . \hfill \\
            \end{aligned}} \right.
\]
Анализируя корни уравнения $\psi_3(\tau)=0$, для различной комбинации
коэффициентов $C_1,C_2,C_3$, получим, что число корней не может быть больше двух. В системе может быть не более двух переключений $u$.

Пусть первое переключение управления происходит в момент времени
$\tau=\tau_1$, а второе в момент времени
$\tau=\tau_2$. Рассмотрим систему \eqref{basesystem} на трех этапах,
при переходе между которыми меняется управление.


\end{frame}


\begin{frame}{Решение задачи быстродействия}
	Этап 1. $u=-u_*$ начальные условия
\[
    m(0)=0;\ \theta(0)=1;\ \omega(0)=\Omega_0;
\]

\[
    \left\{ {\begin{aligned}
                 & 0 = -\tau  u_*+c_1, \hfill                                                            \\
                 & 1 = \frac{1}{2} e^{-\tau } \left(C_1 \left(e^{\tau }-1\right)^2+C_2 \left(e^{2
                \tau }+1\right)+C_3 e^{2 \tau }-C_3+2 e^{\tau } \tau  u_*\right) , \hfill                \\
                 & \Omega_0 = \frac{1}{2} e^{-\tau } \left(C_1 \left(e^{2 \tau }-1\right)+C_2 \left(e^{2
                \tau }-1\right)+C_3 e^{2 \tau }+C_3+2 e^{\tau } u_*\right)  . \hfill                     \\
            \end{aligned}} \right.
\]

\[
    \left\{ {\begin{aligned}
                 & m_1(\tau) = -\tau  u_*, \hfill                                                                         \\
                 & \theta_1(\tau) = \frac{e^\tau+e^{-\tau}}{2}+\frac{\Omega_0-u_*}{2}(e^\tau-e^{-\tau})+\tau u_* , \hfill \\
                 & \omega_1(\tau) =\frac{e^\tau-e^{-\tau}}{2}+\frac{\Omega_0-u_*}{2}(e^\tau+e^{-\tau})+u_*   . \hfill     \\
            \end{aligned}} \right.
\]
Аналогично для 2 и 3 этапов
\end{frame}
\begin{frame}{Решение задачи быстродействия}
	Условие сопряжения этих интервалов
\[
    \left\{ {\begin{aligned}
                 & m_2(\tau_2) = m_3(\tau_2), \hfill            \\
                 & \theta_2(\tau_2) =  \theta_3(\tau_2), \hfill \\
                 & \omega_2(\tau_2) = \omega_3(\tau_2) . \hfill \\
            \end{aligned}} \right.
\]
	Замена переменных
\[
    x=e^{\tau_1} ,\,\,y=e^{\tau_2} ,\,\,z=e^{\frac{\tau_f}{2}}
\]
		\begin{figure}[h!]
			\includegraphics[width=0.8\linewidth]{images/control_intervals.png}
			\caption{Интервалы переключения управления}
		\end{figure}

\end{frame}
	
\begin{frame}{Решение задачи быстродействия}
	Требуется отобрать наименьший корень уравнений больший 1. При различных по знаку $u_*$.
	
		\begin{columns}
			\column{1\textwidth}
			\begin{equation}\label{koshisystem}
				 {\begin{aligned}
					& x=\left( \frac{1}{2z}-\frac{u_*z}{2}-(\Omega_0-u_*)\frac{1}{2z}\right)\frac{z}{u_*(1-z)} \hfill	\\
					& y=zx, \hfill                                                                                              \\
			   \end{aligned}}
			\end{equation}
			\begin{equation}\label{koshisystem}
				\left[ {\begin{aligned}
					& u_*z^2+\Omega _0-1-u_*=0, \hfill    \\
					& (-u_* \Omega _0+u_*^2-u_*)z^4-4 u_*^2z^3+(2 u_* \Omega _0+6 u_*^2-\Omega _0^2+1)z^2- \hfill \\
					& -4 u_*^2z+-u_* \Omega _0+u_*^2+u_*=0                                                                                           \\
			   \end{aligned}} \right.
			\end{equation}
		\end{columns}
		$\tau_f=\ln(z)$

\end{frame}

\begin{frame}{Определение начальных условий для задачи быстродействия}
	Для корректного решения задачи быстродействия необходимо определить начальные условия после толчка.
	
	\hfill \\
	Для этого необходимо построить оценку $\tilde{\eta}$ траектории центра масс системы, зная траекторию центра давления,
	и взять значение $\tilde{\eta_0}$ и $\tilde{\dot{\eta_0}}$ в момент времени завершения толчка

\end{frame}

\begin{frame}{Связь центра масс и центра давления}
		\begin{columns}
		\column{0.5\textwidth}
		\begin{figure}[h!]
			\includegraphics[width=0.7\linewidth]{images_pres/body_1.png}
			\caption{Силы действующие на модель стержня, имитирующего тело человека}
		\end{figure}
		\column{0.5\textwidth}
		\begin{figure}[h!]
			\includegraphics[width=0.8\linewidth]{images_pres/foot.png}
			\caption{Силы действующие на на систему «стопы ног – платформа стабилоанализатора» }
		\end{figure}
	\end{columns}

\end{frame}

\begin{frame}{Связь центра масс и центра давления}
	\begin{columns}
	\column{0.5\textwidth}
	\begin{equation}\label{koshisystem}
    \left\{ {\begin{aligned}
                 & ml\ddot{\theta} = -R_y-F , \hfill   \\
                 & 0=R_z-mg, \hfill \\
                 & J \ddot{\theta} = mlg\theta-Fl_1+M_x . \hfill             \\
            \end{aligned}} \right.
\end{equation}
	\column{0.5\textwidth}
	\begin{equation}\label{koshisystem}
		\left\{ {\begin{aligned}
					 & M_x = Ny+F_yh , \hfill   \\
					 & F_y = R_y , \hfill \\
					 & N \approx mg . \hfill             \\
				\end{aligned}} \right.
	\end{equation}
\end{columns}

	$$M_x=mgy-h\left(F+ml\ddot{\theta}\right)$$

	$$\left(J+mlh\right)\ddot{\theta}=mgl\theta+mgy-Fl_1-Fh$$ 


 	$$\frac{(J+mlh)l\ddot{\theta}}{mgl}=l\theta+y-\frac{F}{mg}(l_1+h);\quad \text{Замена: }\eta=-l\theta; \quad T^2=\frac{J+mlh}{mgl};$$
\begin{equation}\label{eta_y}
	T^2\ddot{\eta}=\eta-y+\frac{F}{mg}(l_1+h)
\end{equation}

\end{frame}

\begin{frame}{Связь центра масс и центра давления}
	Cоотношение \eqref{eta_y} предлагается использовать для определения начальных условий движения сразу после толчка
	
	\hfill \\
	Далее необходимо построить оценку $\tilde\eta$ движения центра масс различными способами, описанными в работах, выполненых под руководством П.А. Кручинина
\end{frame}

\begin{frame}{Моделирование движения человека}
	Модель движения человека, где $M=-C\theta-P\dot\theta$ - момент в голеностопе
		$$J\ddot{\theta}=mgl\theta+M-Fl_1$$
	\begin{columns}
		\column{0.5\textwidth}
		\begin{figure}[h!]
			\includegraphics[width=1\linewidth]{images/pushes_my_1.png}
			\caption{Модель силы толчка}
		\end{figure}
		\column{0.5\textwidth}
		\begin{figure}[h!]
			\includegraphics[width=1\linewidth]{images/deg_my_1.png}
			\caption{Модель изменения угла отклонения}
		\end{figure}
	\end{columns}
\end{frame}

\begin{frame}{Алгоритм фильтрации (композиция двух фильтров)}

Передаточная функция системы (\ref*{eta_y}) имеет вид 
\[
	G(s)=-\frac{1}{T^2s^2-1}
\]
Ее можно представить в виде композиции двух фильтров
\[
	G(s)=G_1(s)\cdot G_2(s) 
\]
\[
	G_1(s)=\frac{1}{Ts-1}, G_2(s)=\frac{1}{Ts+1}
\]
Оценка координаты центра масс может быть найдена, путем последовательного применения двух фильтров
\[
	T \dot{x}+x=-y \text{ в прямом времени}
\]
\[
	T \dot{\eta}-\eta=x \text{ в обратном времени}
\]

\end{frame}

\begin{frame}{Алгоритм фильтрации (преобразование Фурье)}
	$Y(\omega),N(\omega) - \text{Фурье образы y(t) и $\eta$(t)}$
	\[
		N(\omega)=G(i\omega)\cdot Y(\omega)
	\]
	Представим $y(t)=a(t-t_0)+b+\delta(t)$ 

	$a=\frac{y(t_f)-y(t_0)}{t_f-t_0}, b=y(t_0)$, тогда оценка координаты центра масс может быть найдена из

	$\eta(t)=a(t-t_0)+b+\chi(t)$, где $\chi(t)$ - Фурье праобраз $N(\omega)$
	
\end{frame}

\begin{frame}{Моделирование движения человека}
	\begin{figure}[h!]
		\includegraphics[width=0.7\linewidth]{images/cm_my_1.png}
		\caption{Модель изменения саггитальной координаты центра масс и центра давления}
	\end{figure}
\end{frame}
\begin{frame}{Модельная оценка центра масс с использованием FFT}
	\begin{columns}
		\column{0.5\textwidth}
		\begin{figure}[h!]
			\includegraphics[width=1\linewidth]{images/eta_after_FFT.png}
			\caption{Реальное и восстановленное значение $\eta$}
		\end{figure}
		\column{0.5\textwidth}
		\begin{figure}[h!]
			\includegraphics[width=1\linewidth]{images/error_after_fft.png}
			\caption{Ошибка оценивания}
		\end{figure}
	\end{columns}
	$\sigma=0.1mm$
\end{frame}
\begin{frame}{Модельная оценка центра масс с использованием двойной фильтрации}
	\begin{columns}
		\column{0.5\textwidth}
		\begin{figure}[h!]
			\includegraphics[width=1\linewidth]{images/eta_after_filtering.png}
			\caption{Реальное и восстановленное значение $\eta$}
		\end{figure}
		\column{0.5\textwidth}
		\begin{figure}[h!]
			\includegraphics[width=1\linewidth]{images/error_efter_filtering.png}
			\caption{Ошибка оценивания}
		\end{figure}
	\end{columns}
$\sigma=0.008mm$
\end{frame}
\begin{frame}[shrink=7]{Дальнейшие шаги}
	\begin{enumerate}
		\item Применить алгоритм двойной фильтрации и фильтрации через FFT для реальных показаний со стабилоанализатора
		\item Получить оценку ц.м. и оценку скорости изменения ц.м. в момент времени завершения толчка
		\item Определить характерную среднюю скорость изменения момента в голеностопе на участках возврата и подставить ее в управление $u_{\ast}=\frac{\dot{M}_{max}}{mgl\varphi_\ast t_\ast}$
		\item Сравнить реальное время возвращения в вертикальную позу с полученными при решении задачи быстродействия
		\item Построить траекторию центра масс при управленнии, полученном при решении задачи быстродействия
		\item Построить траекторию центра масс при управленнии, полученном при решении задачи быстродействия
		\item Построить траекторию центра масс при управленнии, полученном при решении задачи быстродействия
	  \end{enumerate}
	
\end{frame}

\begin{frame}
	% Please add the following required packages to your document preamble:
% \usepackage{booktabs}
\begin{table}[]
	\centering
	\begin{tabular}{@{}|l|c|c|c|c|@{}}
	\toprule
	\textbf{} &
	  \multicolumn{1}{l|}{\textbf{xStart(сек)}} &
	  \multicolumn{1}{l|}{\textbf{xEnd(сек)}} &
	  \multicolumn{1}{l|}{\textbf{$\Delta Y$(мм)}} &
	  \multicolumn{1}{l|}{\textbf{$\Delta M $(Н $\cdot$ м)}} \\ \midrule
	\textbf{1} & 1098.9 & 1099.3 & 64.7  & 43.1 \\ \midrule
	\textbf{2} & 1120.8 & 1121.0 & 61.5  & 40.9 \\ \midrule
	\textbf{3} & 1133.2 & 1133.6 & 72.6  & 48.3 \\ \midrule
	\textbf{4} & 1185.9 & 1186.2 & 65.18 & 43.4 \\ \midrule
	\textbf{5} & 1277.8 & 1278.0 & 67.3  & 44.7 \\ \bottomrule
	\end{tabular}
	\caption{afafaf}
	\label{tab:my-table}
	\end{table}
\end{frame}


\begin{thebibliography}{15}
	\begin{frame}{Список основной используемой литературы}
		\bibitem{PAKrychinin}П.А. Кручинин Анализ результатов стабилометрических тестов со ступенчатым воздействием с точки зрения механики управляемых систем
		// Биофизика. – 2019. – Т. 64, №5. – С. 1–11.
		\bibitem{mechmodel}П.А. Кручинин Механические модели в стабилометрии // Российский журнал биомеханики. – 2014. – Т. 18, №2. – С. 184–193.
		\bibitem{podoprihin}П.А. Кручинин, М.А. Подоприхин, И.Д. Бекеров Сравнительный анализ алгоритмов оценки движения центра масс по результатам стабилометрических измерений // Биофизика – 2021. – Т. 66, №5. – С. 997–1004. 
		\bibitem{Optimal} Александров В.В., Лемак С.С., Парусников Н.А. Лекции по механике управляемых систем. Москва, Механико-математический факультет МГУ, 2020, 165 с.
		\bibitem{atansfalb}Фалб Питер Л., Атанс Майкл Оптимальное управление, Машиностроение, 1968, 764 с.
	\end{frame}
	$m\ddot{x}=-kx-b\dot{x}-F_{in}=-kx-b\dot{x}-m\ddot{\rho}$
	\begin{frame}{Список основной используемой литературы}
		\bibitem{melnikov}А.А. Мельников, В.В. Филева, М.В. Малахов Эффективность восстановления вертикальной позы
		после толчка у спортсменов разных специализаций // Физиоилогия человека. – 2017. – Т. 43, №4. – С. 78–85.
		\bibitem{melnikov2}А.А. Мельников, В.В. Филева Методика определения устойчивости вертикальной позы под влиянием
		 внешнего толкающего воздействия // Вестник северного (арктического) федерального университета. – 2015. №1. – С. 31–37.
	\end{frame}

\end{thebibliography}

\end{document}